\documentclass{article}
\usepackage[margin=0.5in]{geometry}
\usepackage{xcolor, listings, amsmath, amssymb, tcolorbox}

\definecolor{codegreen}{rgb}{0,0.6,0}
\definecolor{codegray}{rgb}{0.5,0.5,0.5}
\definecolor{codepurple}{rgb}{0.58,0,0.82}

\lstdefinestyle{code}{
	commentstyle=\color{codegreen},
	keywordstyle=\color{magenta},
	numberstyle=\tiny\color{codegray},
	basicstyle=\ttfamily\footnotesize,
	breakatwhitespace=false,
	breaklines=true,
	captionpos=b,
	keepspaces=true,
	numbers=left,
	numbersep=5pt,
	showspaces=false,
	showstringspaces=false,
	showtabs=false,
	tabsize=2
}
\lstset{style=code}

\author{Kyle Guarco}
\title{CS 463: Homework 2}

\begin{document}
\maketitle
\section{Solution}
\textit{Analyze the runtime of the following:}
\lstinputlisting[language=Java, firstline=3, lastline=18]{../MainNoComment.java}

\begin{tcolorbox}
	\centering
	Recall that a repeated sum of ones equals its bound: $\sum_{i = 1}^{n}(1) = n$

	A repeated sum of some constant $C$: $\sum_{i = 1}^{n}(C) = Cn$
\end{tcolorbox}

\noindent We write the loops in terms of summations (the 1 indicates a constant runtime):
\begin{align*}
	\sum_{i = 0}^{N - P}\left( \sum_{j = 0}^{P - 1}{1} \right) &= \sum_{i = 0}^{N - P}{P}  &\mbox{Reduce inner loop using `sum of ones' formula}\\
	\sum_{i = 0}^{N - P}{P} &= P(N - P + 1) &\mbox{Repeated sum of constant $P$}\\
	&= -P^2 + NP + P &\mbox{$\mathbf{O}(NP)$ ignoring the negative term.}\\
\end{align*}
Therefore the runtime of \lstinline[language=Java]{search(String, String)} is $\mathbf{O}(nm)$.

\newpage
\section{Code}
\lstinputlisting[language=Java]{../Main.java}
\end{document}
